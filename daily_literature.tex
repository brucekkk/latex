\documentclass[12pt]{article}
\usepackage[top=0.5cm,bottom=1.4cm,left=1.0cm,right=0.2cm]{geometry}
\usepackage{mathptmx}%
\usepackage{longtable}
\usepackage{array,ragged2e}
\usepackage[table]{xcolor}
\usepackage{caption}
\usepackage{array}
\newcolumntype{L}[1]{>{\raggedright\let\newline\\\arraybackslash\hspace{0pt}}m{#1}}
\newcolumntype{C}[1]{>{\centering\let\newline\\\arraybackslash\hspace{0pt}}m{#1}}
\usepackage{enumitem}
\setlist[enumerate]{itemsep=0mm}
\begin{document}
\listoftables
\clearpage
\begin{longtable}{ C{3cm} | L{16cm} }

%%%%%%%%%%%%%%%%%%%%%%%%%%%%%%%%%%%%%%%%%%%%%%%%%%%%%%%%%%%%
\caption{Asperities and barriers on the seismogenic zone in North Chile: state-of-the-art after the 2007 Mw 7.7 Tocopilla earthquake inferred by GPS and InSAR data}\\ 
%%%%%%%%%%%%%%%%%%%%%%%%%%%%
\hline
\hline
\multicolumn{2}{l}{\textbf{0 Summary}} \\
\hline
\hline 
%\textbf{\textit{Aspects}} & \textbf{\textit{Content}} \\
%\hline 
%\hline
\textit{keyword} & 2007 Tocopilla earthquake\\&111\\ %keyword here

\textcolor{blue}{2007 Tocopilla earthquake} & %content here
\begin{description}[itemsep=-1.5mm]
  \item [General]
  \item [1] ruptured the deeper part of the seismogenic interface (30–50 km) and did not reach the surface 
  \item [2] initiated at the hypocentre and was arrested $\sim$150 km south, beneath the Mejillones Peninsula
  \item [mainshock]
  \item [1]slip concentrated in two main asperities, consistent with previous inversions of seismological data.
  \item [postseismic deformation]
  \item [1]small but still significant post-seismic slip occurred within the first 10 d after the main shock
  \item [2]mostly concentrated at the southern end of the rupture.
  \item [3]represents $\sim$12–19 per cent of the coseismic deformation, of which $\sim$30–55 per cent has been released aseismically.
  \item [slip]
  \item [1]propagated towards relatively shallow depths at its southern extremity, under the Mejillones Peninsula.
  \item [Mejillones Peninsula]
  \item [1]identified as an important structural barrier between two segments of the Peru--Chile subduction zone
  
\end{description}\\

 %%%%%%%%%%%%%%%%%%%%%%%%%%%%
\hline
\hline
\multicolumn{2}{l}{\textbf{1 Research Objective}} \\
\hline
\hline
N/A & N/A  \\ %content here
\textcolor{blue}{N/A} & \textcolor{blue}{All titles have been changed accordingly as follows:}\\
& 3.1 Laboratory Flat Bed Wave Flume Test , \\
& 3.2 Field Scale Flat Bed Test ,\\%content here
\hline    
N/A & N/A  \\%content here
%%%%%%%%%%%%%%%%%%%%%%%%%%%%
\hline
\hline
\multicolumn{2}{l}{\textbf{2 Background and Problems}} \\
\hline
\hline
N/A & N/A  \\ %content here
\textcolor{blue}{N/A} & \textcolor{blue}{All titles have been changed accordingly as follows:}\\
& 3.1 Laboratory Flat Bed Wave Flume Test , \\
& 3.2 Field Scale Flat Bed Test ,\\%content here
\hline    
N/A & N/A  \\%content here
%%%%%%%%%%%%%%%%%%%%%%%%%%%%
\hline
\hline
\multicolumn{2}{l}{\textbf{3 Method}} \\
\hline
\hline
N/A & N/A  \\ %content here
\textcolor{blue}{N/A} & \textcolor{blue}{All titles have been changed accordingly as follows:}\\
& 3.1 Laboratory Flat Bed Wave Flume Test , \\
& 3.2 Field Scale Flat Bed Test ,\\%content here
\hline    
N/A & N/A  \\%content here
%%%%%%%%%%%%%%%%%%%%%%%%%%%%
\hline
\hline
\multicolumn{2}{l}{\textbf{4 Conclusion}} \\
\hline
\hline
N/A & N/A  \\ %content here
\textcolor{blue}{N/A} & \textcolor{blue}{All titles have been changed accordingly as follows:}\\
& 3.1 Laboratory Flat Bed Wave Flume Test , \\
& 3.2 Field Scale Flat Bed Test ,\\%content here
\hline    
N/A & N/A  \\%content here
\hline 
\hline 
%%%%%%%%%%%%%%%%%%%%%%%%%%%%
\end{longtable}
%\end{center}
\end{document}